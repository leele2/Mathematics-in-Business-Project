% !TeX root = ..\main.tex
% Custom Commands
\newcommand\dif{\mathop{}\!\mathrm{d}} %dx for integrals \dif{x}
\newcommand\var[1]{\text{Var}\left[\,#1\,\right]} %Var[]
\newcommand\cov[2]{\text{Cov}\left[\,#1,#2\,\right]} %Cov[x1,x2]
\newcommand\E[1]{\text{E}\left[\,#1\,\right]} %E[]
\newcommand\Ep[2]{\text{E}_{#1}\left[\,#2\,\right]}
\newcommand\IQR{\text{I}\text{Q}\text{R}}
\newcommand{\nline}{\par\vspace{10pt}}
\newcommand\sOneSize{0.60}
\newcommand\sTwoRes{0.70}
\newcommand\PgSz{\scriptsize}
\newcommand\p[1]{\left(#1\right)}
\newcommand\s[1]{\left[#1\right]}
\newcommand\curl[1]{\left\{ #1 \right\}}
%\newcommand{\inc}[3]{#1^{\mathbf{\left(#3 \right)}}_{#2}}
\newcommand{\inc}[3]{#1_{\p{#2,#3}}}
\newcommand{\incm}[3]{\(\inc{#1}{#2}{#3}\)}
\newcommand{\inct}[4]{#1_{\p{#2,#3,#4}}}
\newcommand{\nat}{\mathbb{N}}
\newcommand{\rel}{\mathbb{R}}
\newcommand{\wei}{\mathcal{W}}
\newcommand{\norm}{\mathcal{N}}
\newcommand{\BigO}[1]{\mathcal{O}\left(#1\right)}
\newcommand{\logp}[1]{\log\left(#1 \right)}
\newcommand{\spc}{\phantom{10.0}}
\newcommand{\pof}{\mathbb{V}}
\newcommand{\poff}[1]{\pof\p{#1}}
\newcommandx\nd[3][1=\spc, 2=\spc]{
    \begin{tabular}{cc}
        \multicolumn{2}{c}{#3} \\
        \midrule
        #1 & #2 \\ 
        \bottomrule
    \end{tabular}
}\newcommandx\ndt[4][1=\spc, 2=\spc, 3=\spc]{
  \begin{tabular}{ccc}
      \multicolumn{3}{c}{#4} \\
      \midrule
      #1 & #2 & #3\\ 
      \bottomrule
  \end{tabular}
}
%% Defaults for LstListing Environment
\lstset{
  caption=\lstname,
  basicstyle=\scriptsize\ttfamily, % the size of the fonts that are used for the code
  numbers=left,                   % where to put the line-numbers
  numberstyle=\tiny\color{black},  % the style that is used for the line-numbers
  stepnumber=1,                   % the step between two line-numbers. If it is 1, each line
                                  % will be numbered
  numbersep=5pt,                  % how far the line-numbers are from the code
  backgroundcolor=\color{white},  % choose the background color. You must add \usepackage{color}
  showspaces=false,               % show spaces adding particular underscores
  showstringspaces=false,         % underline spaces within strings
  showtabs=false,                 % show tabs within strings adding particular underscores
  frame=single,                   % adds a frame around the code
  rulecolor=\color{black},        % if not set, the frame-color may be changed on line-breaks within not-black text (e.g. commens (green here))
  tabsize=2,                      % sets default tabsize to 2 spaces
  captionpos=t,                   % sets the caption-position to top
  breaklines=true,                % sets automatic line breaking
  breakatwhitespace=false,        % sets if automatic breaks should only happen at whitespace
  }
%% Defining Style for R Code and Python within LstListing Environment
% !TeX root = ..\main.tex
\lstdefinestyle{R}{ 
  language=R,                     % the language of the code
  keywordstyle=\color{RoyalBlue},      % keyword style
  commentstyle=\color{YellowGreen},   % comment style
  stringstyle=\color{ForestGreen}      % string literal style
}

\lstdefinelanguage{PythonPlus}[]{Python}{
  morekeywords=[1]{,as,assert,nonlocal,with,yield,self,True,False,None,} % Python builtin
  morekeywords=[2]{,__init__,__add__,__mul__,__div__,__sub__,__call__,__getitem__,__setitem__,__eq__,__ne__,__nonzero__,__rmul__,__radd__,__repr__,__str__,__get__,__truediv__,__pow__,__name__,__future__,__all__,}, % magic methods
  morekeywords=[3]{,object,type,isinstance,copy,deepcopy,zip,enumerate,reversed,list,set,len,dict,tuple,range,xrange,append,execfile,real,imag,reduce,str,repr,}, % common functions
  morekeywords=[4]{,Exception,NameError,IndexError,SyntaxError,TypeError,ValueError,OverflowError,ZeroDivisionError,}, % errors
  morekeywords=[5]{,ode,fsolve,sqrt,exp,sin,cos,arctan,arctan2,arccos,pi, array,norm,solve,dot,arange,isscalar,max,sum,flatten,shape,reshape,find,any,all,abs,plot,linspace,legend,quad,polyval,polyfit,hstack,concatenate,vstack,column_stack,empty,zeros,ones,rand,vander,grid,pcolor,eig,eigs,eigvals,svd,qr,tan,det,logspace,roll,min,mean,cumsum,cumprod,diff,vectorize,lstsq,cla,eye,xlabel,ylabel,squeeze,}, % numpy / math
}

\lstdefinelanguage{PyBrIM}[]{PythonPlus}{
  emph={d,E,a,Fc28,Fy,Fu,D,des,supplier,Material,Rectangle,PyElmt},
}

\lstdefinestyle{colored}{ %
  basicstyle=\ttfamily,
  backgroundcolor=\color{white},
  commentstyle=\color{green}\itshape,
  keywordstyle=\color{blue}\bfseries\itshape,
  stringstyle=\color{red},
}

\definecolor{darkred}{rgb}{0.6,0.0,0.0}
\definecolor{darkgreen}{rgb}{0,0.50,0}
\definecolor{lightblue}{rgb}{0.0,0.42,0.91}
\definecolor{orange}{rgb}{0.99,0.48,0.13}
\definecolor{grass}{rgb}{0.18,0.80,0.18}
\definecolor{pink}{rgb}{0.97,0.15,0.45}

\lstdefinestyle{colorEX}{
  basicstyle=\ttfamily,
  backgroundcolor=\color{white},
  commentstyle=\color{darkgreen}\slshape,
  keywordstyle=\color{blue}\bfseries\itshape,
  keywordstyle=[2]\color{blue}\bfseries,
  keywordstyle=[3]\color{grass},
  keywordstyle=[4]\color{red},
  keywordstyle=[5]\color{orange},
  stringstyle=\color{darkred},
  emphstyle=\color{pink}\underbar,
}
%% Hyperlink customization
\hypersetup{
  frenchlinks  = false, %Small caps for hyperlinks
  colorlinks   = true, %Colours links instead of boxes
  urlcolor     = blue, %Colour for external hyperlinks
  linkcolor    = black, %Colour of internal links
  citecolor   = black   %Colour of citations
}
%% Extra spacing on matrices
\makeatletter
\renewcommand*\env@matrix[1][\arraystretch]{%
  \edef\arraystretch{#1}%
  \hskip -\arraycolsep{}
  \let{}\@ifnextchar\new@ifnextchar{}
  \array{*\c@MaxMatrixCols{} c}}
\makeatother
%%Capitalising \autoref
\renewcommand*{\chapterautorefname}{Chapter}
\renewcommand*{\subsectionautorefname}{Subsection}
\renewcommand*{\sectionautorefname}{Section}
\renewcommand*{\itemautorefname}{property}
%%Merging list of tables and figures
\def\table{\def\figurename{Table}\figure}
\let\endtable\endfigure{}
\renewcommand\listfigurename{List of Figures and Tables}
%%Formating Chapter spacing and linebreak
\titleformat{\chapter}[hang]
{\normalfont\huge\bfseries}{\chaptertitlename\ \thechapter:}{1em}{}
\titlespacing*{\chapter}{0pt}{0pt}{5pt}